%----------------------------------------------------------------------------------------
%	PACKAGES AND OTHER DOCUMENT CONFIGURATIONS
%----------------------------------------------------------------------------------------

\documentclass[12pt]{report}
\usepackage[spanish]{babel}
\usepackage[latin1]{inputenc}
\usepackage{amsmath}
\usepackage{graphicx}
\usepackage{listings}   
\usepackage{hyperref}
\usepackage{color}
\usepackage{pdfpages}



% Default fixed font does not support bold face
\DeclareFixedFont{\ttb}{T1}{txtt}{bx}{n}{12} % for bold
\DeclareFixedFont{\ttm}{T1}{txtt}{m}{n}{12}  % for normal

% Custom colors
\usepackage{color}
\definecolor{deepblue}{rgb}{0,0,0.5}
\definecolor{deepred}{rgb}{0.6,0,0}
\definecolor{deepgreen}{rgb}{0,0.5,0}

\usepackage{listings}
\newcommand\pythonstyle{\lstset{
language=Python,
basicstyle={\small\ttfamily},
aboveskip=3mm,
columns=flexible,
breaklines=true,
breakatwhitespace=true,
tabsize=3,
belowskip=3mm,
otherkeywords={self},             % Add keywords here
keywordstyle=\ttfamily\color{deepblue},
emph={MyClass,__init__},          % Custom highlighting
emphstyle=\ttfamily\color{deepred},    % Custom highlighting style
stringstyle=\color{deepgreen},
frame=tb,                         % Any extra options here
showstringspaces=false            % 
}}

\lstnewenvironment{python}[1][]
{
\pythonstyle
\lstset{#1}
}
{}


\begin{document}

\begin{titlepage}

\newcommand{\HRule}{\rule{\linewidth}{0.5mm}} % Defines a new command for the horizontal lines, change thickness here

\center % Center everything on the page
 

\textsc{\LARGE Universidad de Buenos Aires}\\[1.5cm] % Name of your university/college
\textsc{\Large TP Python}\\[0.5cm] % Major heading such as course name
\textsc{\large 2015-2016}\\[0.5cm] % Minor heading such as course title


\HRule \\[0.4cm]
{ \huge \bfseries Python NetHack}\\[0.4cm] % Title of your document
\HRule \\[1.5cm]
 

\begin{minipage}{0.4\textwidth}
\begin{center}
Nicolas \textsc{Slimmens}\\
Jean \textsc{Guis}\\
\end{center}
\end{minipage}

\bigskip


{\large 9 de noviembre 2015}\\[4cm] % Date, change the \today to a set date if you want to be precise

%----------------------------------------------------------------------------------------
%	LOGO SECTION
%----------------------------------------------------------------------------------------

\includegraphics[scale=1]{images/uba.png}\\[1cm] % Include a department/university logo - this will require the graphicx package
 
%----------------------------------------------------------------------------------------

\vfill % Fill the rest of the page with whitespace

\end{titlepage}


\tableofcontents
\chapter{NetHack}

\section{Section 1.1}
Lorem ipsum dolor sit amet, consectetur adipiscing elit. Integer commodo sit amet turpis sit amet ullamcorper. Nulla non arcu nec ligula dictum rhoncus. Nulla vel ornare libero. Fusce porttitor ultrices dictum. Etiam pharetra id arcu id suscipit. Donec eu massa a mi molestie ultrices. Suspendisse potenti. 
\subsection{Subsection 1.1.1}
Fusce dapibus, velit ut viverra ornare, diam augue interdum sem, vel porta orci ante in augue. Integer eget consectetur dui. Sed ultricies turpis vel porttitor condimentum. Class aptent taciti sociosqu ad litora torquent per conubia nostra, per inceptos himenaeos. Nullam malesuada, nulla eget vestibulum porttitor, ex magna rutrum sem, sit amet vehicula lectus nibh posuere dolor. Suspendisse hendrerit dolor est, sed ultricies leo facilisis nec. Mauris egestas cursus quam, in scelerisque arcu vehicula vel. In lacus velit, semper id maximus eu, consequat eget libero. Praesent eros massa, aliquam eget risus quis, vestibulum posuere nisl. Donec iaculis elit eget accumsan congue. Pellentesque vel elit sit amet odio volutpat tincidunt ac quis tellus. Integer non luctus augue. Donec ut ultricies urna, ac euismod quam. Sed eget condimentum nisi, vel posuere ante. Nunc blandit pellentesque eros. \\\\
Sed a tortor et ante scelerisque viverra. Mauris viverra enim in sodales dapibus. Phasellus felis lectus, commodo sit amet tempor congue, ornare ac nisl. Duis at dui at libero imperdiet hendrerit ac a nisl. Quisque non pharetra metus. Etiam faucibus neque sed molestie luctus. Proin ut sem nisl. Donec et urna sit amet turpis posuere tempus id bibendum lorem. Proin et ante et est molestie porta eu eu ex. Nullam interdum iaculis nunc. Cras nec enim lorem. 
\subsubsection{Python example}
Aliquam a feugiat arcu. Lorem ipsum dolor sit amet, consectetur adipiscing elit. Donec maximus, lorem quis bibendum hendrerit, massa arcu hendrerit sapien, vitae semper leo turpis sed erat. Praesent nunc eros, vulputate eu mi nec, porttitor malesuada enim. Donec maximus imperdiet mauris, vel consequat diam faucibus non. Nam pretium quam mauris, et facilisis est congue suscipit. Donec eu interdum diam. Pellentesque elementum elit quis condimentum consectetur. Nulla facilisi. Fusce ut urna ac neque feugiat ornare non vel ipsum. Vivamus consequat, risus vitae pulvinar feugiat, turpis ante iaculis ligula, vel elementum felis leo a metus. Nullam pharetra congue arcu vel tempus. 
\begin{python}
		minLeft = min(self.getGridDimension()[1] - lastPlayed[0]-2, lastPlayed[1]) 
		minRight =  min(lastPlayed[0]+1, self.getGridDimension()[0]-lastPlayed[1]-1)

		xstart = lastPlayed[1] - minLeft # Starting x coordinate
		ystart = lastPlayed[0] + minLeft + 1 # Starting y coordinate

		if(self.getGridDimension()[0] - xstart >= 4 and ystart >= 4): 
			result = []
			nbCase =  min(ystart - 3 , self.getGridDimension()[0] - (xstart +3))
			for i in xrange(0,nbCase):
				result.append([self.grid[ystart-i][xstart+i], self.grid[ystart-1-i][xstart+1+i],
					self.grid[ystart-2-i][xstart+2+i],self.grid[ystart-3-i][xstart+3+i]])
			if [player for t in range(0,4)] in result:
				return player
\end{python}


\chapter{El c�digo}


\begin{thebibliography}{9}
\bibitem{latexcompanion} 
Mark Lutz,
\textit{Learning Python}. 
O'Reilly, Fourth edition, ISBN: 978-0-596-15806-4, September 2009.

\end{thebibliography}


\end{document}